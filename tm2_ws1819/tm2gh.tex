\documentclass[11pt]{article}
\usepackage[ngerman]{babel}

\usepackage{amsmath,amssymb, a4, verbatim}
%\usepackage[latin1]{inputenc}
\usepackage[utf8]{inputenc} % üöäß
\usepackage{listings} % für inline codelistings
\lstset{%
		basicstyle=\ttfamily,		% the size of the fonts
		columns=fixed,						% anything else is horrifying
		showspaces=false,				% show spaces using underscores?
		showstringspaces=false,		% underline spaces within strings?
		showtabs=false,						% show tabs within strings?
		xleftmargin=1.5em,				% left margin space
}
\lstdefinestyle{inline}{basicstyle=\ttfamily}
\newcommand{\listline}[1]{\lstinline[style=inline]!#1!}

\usepackage{caption}
\newcommand{\tinycaption}[1]{\captionsetup{labelformat=empty}\caption{#1}}

%\usepackage{color}
%\usepackage{epsfig} % eps
\usepackage{graphicx} % eps
%\usepackage[shortcuts]{extdash}
%\usepackage{dsfont}
%\usepackage{epstopdf} % eps
%\usepackage[pdf]{pstricks} % eps
%\usepackage{auto-pst-pdf}
\usepackage{mathtools}
\usepackage{dsfont} % $ \mathds{1} $
\usepackage{icomma}
\usepackage{tikz}
%\usepackage{pgfplots}

%\pgfplotsset{compat=1.8}
\usepackage[bottom]{footmisc} % put footnotes at the bottom of page
\usepackage{nicefrac} % für brüche die aussehen wie prozentzeichen
% \usepackage{ps2pdf}
\usetikzlibrary{automata,positioning}

\usepackage{algorithmicx}
\usepackage{algpseudocode}
\usepackage{algorithm}

\usepackage{multicol}
\usepackage{wrapfig} % make stuff float
\usepackage{placeins} % stop stuff from floating
\usepackage{seqsplit} % very long numbers
\usepackage{framed} % begin{framed}

%	Headings and Footings :
\usepackage{fancyhdr}
\headheight15pt
\lhead{Technische Mechanik II, \ueberschrift}

\chead{}
\usepackage{lastpage}
\rhead{Seite \thepage~von \pageref{LastPage}}
\renewcommand{\headrulewidth}{.4pt}

\usepackage[yyyymmdd]{datetime}
\renewcommand{\dateseparator}{--}
\lfoot{\today}
\cfoot{}
\rfoot{Joshua}
\renewcommand{\footrulewidth}{.4pt}

%----------------------------------------------------------------

\textwidth16.5cm
\oddsidemargin0.cm
\evensidemargin0.cm

\def\somedistanceTop{2cm}
\def\somedistanceLeft{1cm}

\usepackage[]{geometry}
\geometry{
	a5paper,
	%	total={170mm,257mm},
	left=\somedistanceLeft,
	right=\somedistanceLeft,
	top=\somedistanceTop,
	bottom=\somedistanceTop
}


\parindent0cm


\newcommand{\R}{ {\mathbb R} }
\newcommand{\C}{ {\mathbb C} }
\newcommand{\1}{ {\mathds{1}} }
\newcommand{\abs}[1]{\lvert#1\rvert}
\newcommand{\norm}[1]{\left\lVert#1\right\rVert}
\newcommand{\xt}{\tilde{x}}
\newcommand{\dotleq}{\dot{\leq}}
\newcommand{\m}{\hphantom{-} }

\newcommand{\dashfill}[1]{\vspace{11pt}\def\dashfill{\cleaders\hbox{#1}\hfill}\hbox to \hsize{\dashfill\hfil}\vspace{11pt}}
\newcommand{\scdot}{\!\cdot\!}


\newcommand{\sig}{\text{signum}}
\newcommand{\rot}{}

% ------------------	edit Ueberschrift ---------------------
\newcommand{\ueberschrift}{TM II Formeln}


\usepackage{mathtools}
\usepackage{ragged2e}
\newlength\ubwidth
\newlength\obwidth
\newcommand\underbraceWrap[3][0pt]
{
	\settowidth\ubwidth{$#1$}
	\underbrace{#2}_
	{
		\parbox
			{
				\maxof{\ubwidth}{\numexpr#1}
			}
			{
				\scriptsize\Centering#3
			}
	}
}
\newcommand\overbraceWrap[3][0pt]
{
	\settowidth\obwidth{$#1$}
	\overbrace{#2}^
		{
			\parbox
				{
					\maxof{\obwidth}{\numexpr#1}
				}
				{
					\scriptsize\Centering#3
				}
		}
}

%for use in \int
\newcommand{\td}{\,\text{d}}

\input{tex/greek.tex}

\setcounter{secnumdepth}{5}
\setcounter{tocdepth}{5}

% for clickabe stuffs
\usepackage{hyperref}
\hypersetup{
	colorlinks,
	citecolor=black,
	filecolor=black,
	linkcolor=black,
	urlcolor=black
}


% -----------------------------------------------------------
\begin{document}
		\abovedisplayskip = 5pt plus 40pt minus 5pt
		\belowdisplayskip = 5pt plus 40pt minus 5pt
		\abovedisplayshortskip = 2pt plus 40pt minus 2pt
		\belowdisplayshortskip = 2pt plus 40pt minus 2pt
		\pagestyle{fancy}
		
%	\begin{multicols}{2}
		\pagenumbering{roman}
		\tableofcontents
%	\end{multicols}
		\pagebreak

		\pagenumbering{arabic}

		\section{Zug und Druck in Stäben}
		\subsection{Spannung} % 1.1


		\begin{equation}
			\boxed{
				\tag{1.1}
				\begin{split}
					\underbrace{\sigma}_{\text{Spannung}\left[\frac{N}{mm^2}\right]}
					=
					\frac{\overbrace{N}^{\text{Normalspannung}\left[N\right]}}{\underbrace{A}_{\text{Fläche}\left[mm^2\right]}}
				\end{split}
			}
		\end{equation}

		\begin{equation}
			\boxed{
				\tag{1.2}
				\begin{split}
					\underbrace{\sigma}_{\text{Spannung}\left[\frac{N}{mm^2}\right]}
					=
					\frac{\overbrace{F}^{\text{Kraft}\left[N\right]}}{\underbrace{A}_{\text{Fläche}\left[mm^2\right]}}
				\end{split}
			}
		\end{equation}

		\begin{equation}
			\boxed{
				\tag{1.3}
				\begin{split}
					\sigma
					=
					\frac{\overbraceWrap[95pt]{\sigma_0}{Normalspannung in einem Schnitt Senkrecht zur Stabachse}}{2}
					\left(1 + \cos 2 \varphi\right)
					,&
					\tau
					=
					\frac{\sigma_0}{2}
					\left(\sin 2 \varphi\right)
				\end{split}
			}
		\end{equation}

		\begin{equation}
			\boxed{
				\tag{1.4}
				\begin{split}
					\sigma(x)
					=
					\frac{N(x)}{A(x)}
				\end{split}
			}
		\end{equation}

		\begin{equation}
			\boxed{
				\tag{1.5}
				\begin{split}
					A_{\text{erf}}
					=
					\frac{\abs{N}}{\sigma_{\text{zul}}}
				\end{split}
			}
		\end{equation}

		\subsection{Dehnung} % 1.2

		\begin{equation}
			\boxed{
				\tag{1.6}
				\begin{split}
					\underbrace{\varepsilon}_{\text{Dehnung} \left[1\right]} 
					=
					\frac{\overbrace{\Delta \ell}^{\text{Verlängerung}\left[m\right]}}
							 {\underbraceWrap[50pt]{\ell_0}{Ursprüngliche Länge $\left[m\right]$}}
					=
					\frac{\ell - \ell_0}{\ell_0}
				\end{split}
			}
		\end{equation}


		Örtliche (lokale Dehnung)
		\begin{equation}
			\boxed{
				\tag{1.7}
				\begin{split}
					\varepsilon(x)
					=
					\frac{\text{d}u}
							 {\text{d}x}
				\end{split}
			}
		\end{equation}


		\subsection{Stoffgesetz}

		Hooke'sches Gesetz
		\begin{equation}
			\boxed{
				\tag{1.8}
				\begin{split}
					\underbraceWrap[60pt]{E}{Elastizitätsmodul $\left[\frac{N}{mm^2}\right]$}
					=
					\frac{\overbrace{\sigma}^{\text{Spannung}\left[\frac{N}{mm^2}\right]}}
							 {\underbrace{\varepsilon}_{\text{Dehnung}\left[1\right]}}
				\end{split}
			}
		\end{equation}

		Umgestellt nach Sigma, übliche Form:
		\begin{equation*}
				\begin{split}
					\sigma
					=
					E \varepsilon
					=
					\frac{\Delta \ell}{\ell_0} E
				\end{split}
		\end{equation*}

		\begin{equation}
			\boxed{
				\tag{1.9}
				\begin{split}
					\underbrace{\varepsilon}_{\text{Dehnung} \left[1\right]}
					=
					\frac{\sigma}{E}
				\end{split}
			}
		\end{equation}

		\begin{equation}
			\boxed{
				\tag{1.10}
				\begin{split}
					\underbrace{\varepsilon_T}_{\text{Wärmedehnung} \left[1\right]}
					=
					\underbraceWrap[110pt]
						{\alpha}
						{
							Thermischer Ausdehnungskoeffizient (Wärmeausdehnugnskoeffizient)
							$\left[
								\nicefrac{1}
												 {\,^{\circ}\mathrm{C}}
							\right]$
						}
					\cdot
					\underbrace{\Delta T}_{\text{Temperaturänderung}\left[\,^{\circ}\mathrm{C}\right]}
				\end{split}
			}
		\end{equation}

		\begin{equation}
			\boxed{
				\tag{1.11}
				\begin{split}
					\varepsilon
					=
					\frac{\sigma}{E}
					+
					\alpha_T
					\Delta T
				\end{split}
			}
		\end{equation}
		
		\begin{equation}
			\boxed{
				\tag{1.12}
				\begin{split}
					\sigma
					=
					E
					\left(
						\varepsilon
						-
						\alpha_T
						\Delta T
					\right)
				\end{split}
			}
		\end{equation}


		\subsection{Einzelstab}

		\begin{equation}
			\boxed{
				\tag{1.13}
				\begin{split}
					\frac{\text{d}N}
							 {\text{d}x}
					+
					\underbrace{n}_{\text{Linienkraft}}
					=
					0
				\end{split}
			}
		\end{equation}

		\begin{equation}
			\boxed{
				\tag{1.14}
				\begin{split}
					\frac{\text{d}u}
							 {\text{d}x}
					=
					\frac{N}{EA}
					+
					\alpha_T \Delta T
				\end{split}
			}
		\end{equation}

		\begin{equation}
			\boxed{
				\tag{1.15}
				\begin{split}
					\Delta \ell
					=
					u(l)
					-
					u(0)
					=
					\int_{0}^{\ell}
					\varepsilon
					\text{d}x
				\end{split}
			}
		\end{equation}
		 
		\begin{equation}
			\boxed{
				\tag{1.16}
				\begin{split}
					\Delta \ell
					=
					\int_{0}^{\ell}
					\left(
						\frac{N}{EA}
						+
						\alpha_T \Delta T
					\right)
					\text{d}x
				\end{split}
			}
		\end{equation}

		\begin{equation}
			\boxed{
				\tag{1.17}
				\begin{split}
					\Delta \ell
					=
					\frac{F\ell}{EA}
					+
					\alpha_T \Delta T \ell
				\end{split}
			}
		\end{equation}


		Für $\Delta T = 0$
		\begin{equation}
			\boxed{
				\tag{1.18}
				\begin{split}
					\Delta \ell
					=
					\frac{F\ell}{EA}
				\end{split}
			}
		\end{equation}

		Oder $F = 0$
		\begin{equation}
			\boxed{
				\tag{1.19}
				\begin{split}
					\Delta \ell
					=
					\alpha_T \Delta T \ell
				\end{split}
			}
		\end{equation}
		
		\begin{equation}
			\boxed{
				\tag{1.20a}
				\begin{split}
					\left(
						EA u'
					\right)'
					=
					-n
					+
					\left(
						EA \alpha_t \Delta T
					\right)'
				\end{split}
			}
		\end{equation}

		Sei in 1.20a $EA = const$ und $\Delta T = const$
		\begin{equation}
			\boxed{
				\tag{1.20b}
				\begin{split}
					EA u''
					=
					-n
				\end{split}
			}
		\end{equation}


		\subsection{Statisch bestimmte Stabsysteme}

		\begin{equation}
			\boxed{
				\tag{1.21}
				\begin{split}
					u
					&=
					\abs{\Delta\ell_1}
					=
					\frac{F\ell}{EA}
					\frac{1}{\tan{\alpha}}
					,\\
					v
					&=
					\frac{\Delta\ell_2}{\sin{\alpha}}
					+
					\frac{u}{\tan{\alpha}}
					=
					\frac{F\ell}{EA}
					\frac{1 + \cos^3\alpha}{\sin^2\alpha \cos \alpha}
				\end{split}
			}
		\end{equation}


		\subsection{Statisch unbestimmte Stabsysteme}

		\subsection{Zusammenfassung}
		
		\section{Spannungszustand}
		\subsection{Spannungvektor und Spannungtensor}

		\begin{equation}
			\boxed{
				\tag{2.1}
				\begin{split}
					\boldsymbol{t}
					=
					\lim_{\Delta A \rightarrow 0}
					\frac{\Delta \boldsymbol{F}}{\Delta A}
					=
					\frac{\text{d}\boldsymbol{F}}{\text{d}A}
				\end{split}
			}
		\end{equation}

		\begin{equation}
			\boxed{
				\tag{2.2}
				\begin{split}
					\boldsymbol{t}
					=
					\tau_{yx} \boldsymbol{e_x}
					+
					\sigma_y	\boldsymbol{e_y}
					+
					\tau_{yz} \boldsymbol{e_z}
				\end{split}
			}
		\end{equation}

		\begin{equation}
			\boxed{
				\tag{2.3}
				\begin{split}
					\tau_{xy} = \tau_{yx},
					\tau_{xz} = \tau_{zx},
					\tau_{yz} = \tau_{zy}
				\end{split}
			}
		\end{equation}
		
		\begin{equation}
			\boxed{
				\tag{2.4}
				\begin{split}
					\boldsymbol{\sigma}
					=
					\begin{bmatrix*}
						\sigma_{x} & \tau_{xy}	& \tau_{xz} \\
						\tau_{yx}	& \sigma_{y} & \tau_{yz} \\
						\tau_{zx}	& \tau_{zy}	& \sigma_{z}
					\end{bmatrix*}
					=
					\begin{bmatrix*}
						\sigma_{x} & \tau_{xy}	& \tau_{xz} \\
						\tau_{xy}	& \sigma_{y} & \tau_{yz} \\
						\tau_{xz}	& \tau_{yz}	& \sigma_{z}
					\end{bmatrix*}					
				\end{split}
			}
		\end{equation}

		\subsection{Ebener Spannungszustand}
		
		\subsubsection{Koordinatentransformation}

		\begin{equation}
			\boxed{
				\tag{2.5a}
				\begin{split}
					\sigma_{\xi} &= \sigma_x \cos^2 \varphi + \sigma_y \sin^2\varphi + 2 \tau_{xy} \sin\varphi \cos\varphi\\
					\tau_{\xi \eta} &= -(\sigma_x - \sigma_y) \sin \varphi \cos \varphi + \tau_{xy} (\cos^2\varphi - \sin^2\varphi)
				\end{split}
			}
		\end{equation}
	 
		\begin{equation}
			\boxed{
				\tag{2.5b}
				\begin{split}
					\sigma_{\eta}
					=
					\sigma_{x} \sin^2 \varphi + \sigma_y \cos^2 \varphi - 2 \tau_{xy} \cos\varphi\sin\varphi
				\end{split}
			}
		\end{equation}

		\begin{equation}
			\boxed{
				\tag{2.6}
				\begin{split}
					\sigma_{\xi} &= \frac{1}{2} (\sigma_x + \sigma_y) +&\frac{1}{2}(\sigma_x - \sigma_y) \cos 2 \varphi + \tau_{xy} \sin 2 \varphi, \\
					\sigma_{\eta} &= \frac{1}{2} (\sigma_x + \sigma_y) -&\frac{1}{2}(\sigma_x - \sigma_y) \cos 2 \varphi + \tau_{xy} \sin 2 \varphi,\\
					\tau_{\xi \eta} &= -&\frac{1}{2}(\sigma_x - \sigma_y) \sin 2 \varphi + \tau_{xy} \cos 2 \varphi,
				\end{split}
			}
		\end{equation}
		
		\begin{equation}
			\boxed{
				\tag{2.7}
				\begin{split}
					\sigma_\xi + \sigma_\eta
					=
					\sigma_x + \sigma_y
				\end{split}
			}
		\end{equation}

		\subsubsection{Hauptspannungen}
		
		\begin{equation}
			\boxed{
				\tag{2.8}
				\begin{split}
					\tan 2\varphi^\ast
					=
					\frac{2 \tau_{xy}}{\sigma_x - \sigma_y}
				\end{split}
			}
		\end{equation}

		\begin{equation}
			\boxed{
				\tag{2.9}
				\begin{split}
					\cos 2\varphi^\ast
					&=
					\frac{1}{\sqrt{1+ \tan^2 2\varphi^\ast}}
					&=
					\frac{\sigma_x - \sigma_y}{\sqrt{(\sigma_x - \sigma_y)^2 + 4 \tau_{xy}^2}} \\
					\sin 2\varphi^\ast
					&=
					\frac{\tan 2\varphi^\ast}{\sqrt{1+ \tan^2 2\varphi^\ast}}
					&=
					\frac{2\tau_{xy}}{\sqrt{(\sigma_x - \sigma_y)^2 + 4 \tau_{xy}^2}} \\					
				\end{split}
			}
		\end{equation}

		\begin{equation}
			\boxed{
				\tag{2.10}
				\begin{split}
					\sigma_{1,2}
					=
					\frac{\sigma_x + \sigma_y}{2}
					\pm
					\sqrt{\left(
						\frac{\sigma_x - \sigma_y}{2}
						\right)^2
						+
						\tau_{xy}^2
					}
				\end{split}
			}
		\end{equation}

		\begin{equation}
			\boxed{
				\tag{2.11}
				\begin{split}
					\tan 2\varphi^{\ast\ast}
					=
					-\frac{\sigma_x - \sigma_y}{2 \tau_{xy}}
				\end{split}
			}
		\end{equation}

		\begin{equation}
			\boxed{
				\tag{2.12a}
				\begin{split}
					\tau_{\text{max}}
					=
					\pm
					\sqrt{(\frac{\sigma_x - \sigma_y}{2})^2 + \tau_{xy}^2}
				\end{split}
			}
		\end{equation}

		\begin{equation}
			\boxed{
				\tag{2.12b}
				\begin{split}
					\tau_{\text{max}}
					=
					\pm
					\frac{1}{2}(\sigma_1 -\sigma_2)
				\end{split}
			}
		\end{equation}

		\begin{equation}
			\boxed{
				\tag{2.13}
				\begin{split}
					\sigma_M
					=
					\frac{1}{2}(\sigma_x + \sigma_y)
					=
					\frac{1}{2}(\sigma_1 + \sigma_2)
				\end{split}
			}
		\end{equation}

		\subsection{Mohrscher Spannungkreis}

		\begin{equation}
			\boxed{
				\tag{2.14}
				\begin{split}
					\sigma_\xi-\frac{1}{2}(\sigma_x + \sigma_y)
					&=
					\frac{1}{2}(\sigma_x -\sigma_y)\cos 2\varphi + \tau_{xy}\cos 2 \varphi\\
					\tau_{\xi\eta}
					&=
					- \frac{1}{2}(\sigma_x - \sigma_y) \sin 2 \varphi + \tau_{xy} \cos 2\varphi
				\end{split}
			}
		\end{equation}		

		\begin{equation}
			\boxed{
				\tag{2.15}
				\begin{split}
					\left[\sigma_\xi - \frac{1}{2} (\sigma_x + \sigma_y)\right]^2
					+
					\tau_{\xi \eta}^2
					=
					\left(
						\frac{\sigma_x - \sigma_y}{2}
					\right)^2
					+
					\tau_{xy}^2
				\end{split}
			}
		\end{equation}		

		\begin{equation}
			\boxed{
				\tag{2.16}
				\begin{split}
					\left(\sigma - \sigma_M\right)^2
					+
					\tau^2
					=
					r^2
				\end{split}
			}
		\end{equation}		

		\begin{equation}
			\boxed{
				\tag{2.17}
				\begin{split}
					r^2
					=
					\frac{1}{4}
					\left[
						(\sigma_x + \sigma_y)^2
						-
						4 (\sigma_x\sigma_y -\tau_{xy}^2)
					\right]
				\end{split}
			}
		\end{equation}		

		\subsubsection{Dünnwandiger Kessel}

		\begin{equation}
			\boxed{
				\tag{2.18}
				\begin{split}
					\sigma_x
					=
					\frac{1}{2} \,p\, \frac{r}{t}
				\end{split}
			}
		\end{equation}

		\begin{equation}
			\boxed{
				\tag{2.19}
				\begin{split}
					\sigma_{\varphi}
					=
					p\,
					\frac{r}{t}
				\end{split}
			}
		\end{equation}

		\begin{equation}
			\boxed{
				\tag{2.20}
				\begin{split}
					\sigma_t
					=
					\sigma_\varphi
					=
					\frac{1}{2}\,p\,\frac{r}{t}
				\end{split}
			}
		\end{equation}

		\subsection{Gleichgewichtsbedingungen}

		\begin{equation}
			\boxed{
				\tag{2.21a}
				\begin{split}
					\frac{\partial\sigma_x}{\partial x}
					+
					\frac{\partial\tau_{yx}}{\partial y}
					+
					f_x
					=
					0
				\end{split}
			}
		\end{equation}

		\begin{equation}
			\boxed{
				\tag{2.21b}
				\begin{split}
					\frac{\partial\tau_{xy}}{\partial x}
					+
					\frac{\partial\sigma_{y}}{\partial y}
					+
					f_y
					=
					0
				\end{split}
			}
		\end{equation}

		\begin{equation}
			\boxed{
				\tag{2.22}
				\begin{split}
					\frac{\partial\sigma_{x}}{\partial x}
					+
					&\frac{\partial\tau_{yx}}{\partial y}
					+
					\frac{\partial\tau_{zx}}{\partial z}
					+
					f_x\!&= 0 \\
					\frac{\partial\tau_{xy}}{\partial x}
					+
					&\frac{\partial\sigma_{y}}{\partial y}
					+
					\frac{\partial\tau_{zy}}{\partial z}
					+
					f_y\!&= 0 \\
					\frac{\partial\tau_{xz}}{\partial x}
					+
					&\frac{\partial\tau_{yz}}{\partial y}
					+
					\frac{\partial\sigma_{z}}{\partial z}
					+
					f_z\!&= 0 \\
				\end{split}
			}
		\end{equation}		

		\subsection{Zusammenfassung}
		
		\section{Verzerrungszustand, Elastizitätsgesetze}
		\subsection{Verzerrungszustand}

		\begin{equation}
			\boxed{
				\tag{3.1}
				\begin{split}
					\varepsilon_x
					=
					\frac{\partial u}{\partial x}, \;\;
					\varepsilon_y
					=
					\frac{\partial v}{\partial y}				 
				\end{split}
			}
		\end{equation}		

		\begin{equation}
			\boxed{
				\tag{3.2}
				\begin{split}
					\gamma_{xy}
					=
					\frac{\partial u}{\partial y}
					+
					\frac{\partial v}{\partial x}
				\end{split}
			}
		\end{equation}		

		\begin{equation}
			\boxed{
				\tag{3.3}
				\begin{split}
					\text{da bin ich jetzt zu faul}
				\end{split}
			}
		\end{equation}		

		\begin{equation}
			\boxed{
				\tag{3.4}
				\begin{split}
					\tan 2\varphi^\ast
					=
					\frac{\gamma_{xy}}{\varepsilon_x -\varepsilon_y}
				\end{split}
			}
		\end{equation}

		\begin{equation}
			\boxed{
				\tag{3.5}
				\begin{split}
					\epsilon_{1,2} = \frac{\epsilon_x + \epsilon_y}{2}
					\pm
					\sqrt{\left( \frac{\epsilon_x - \epsilon_y}{2}\right) + \left( \frac{1}{2} \gamma_{xy} \right) }
				\end{split}
			}
		\end{equation}		

		\begin{equation}
			\boxed{
				\tag{3.6a}
				\begin{split}
					\epsilon_x = \frac{\partial u}{\partial x}, \quad
					\epsilon_y = \frac{\partial v}{\partial y}, \quad
					\epsilon_z = \frac{\partial w}{\partial z}, \quad
				\end{split}
			}
		\end{equation}		

		\begin{equation}
			\boxed{
				\tag{3.6b}
				\begin{split}
					\gamma_{xy} = \frac{\partial u}{\partial y} + \frac{\partial v}{\partial x}, \quad
					\gamma_{xz} = \frac{\partial u}{\partial z} + \frac{\partial w}{\partial x}, \quad
					\gamma_{yz} = \frac{\partial v}{\partial z} + \frac{\partial w}{\partial y}, \quad
				\end{split}
			}
		\end{equation}		

		\begin{equation}
			\boxed{
				\tag{3.7}
				\begin{split}
					\mathbf{\epsilon}
					=
					\begin{bmatrix}
						\epsilon_{x } & \epsilon_{xy} & \epsilon_{xz} \\
						\epsilon_{yx} & \epsilon_{y } & \epsilon_{yz} \\
						\epsilon_{zx} & \epsilon_{zy} & \epsilon_{z } 
					\end{bmatrix}
					=
					\begin{bmatrix}
						\epsilon_x & \frac{1}{2}\gamma_{xy} & \frac{1}{2}\gamma_{xz}\\
						\frac{1}{2}\gamma_{xy} & \epsilon_x & \frac{1}{2}\gamma_{yz}\\
						\frac{1}{2}\gamma_{xz} & \frac{1}{2}\gamma_{yz} & \epsilon_z
					\end{bmatrix}
				\end{split}
			}
		\end{equation}

		\subsection{Elastizitätsgesetz}

		\begin{equation}
			\boxed{
				\tag{3.8}
				\begin{split}
			\varepsilon_y = -\nu \varepsilon_x
				\end{split}
			}
		\end{equation}

		\begin{equation}
			\boxed{
				\tag{3.9}
				\begin{split}
			\varepsilon_x = \frac{1}{E} \left( \sigma_x - \nu\sigma_y \right), 
			\varepsilon_y = \frac{1}{E} \left( \sigma_y - \nu\sigma_x \right)
				\end{split}
			}
		\end{equation}

		\begin{equation}
			\boxed{
				\tag{3.10}
				\begin{split}
			\tau_{xy} = G \gamma_{xy}
				\end{split}
			}
		\end{equation}

		\begin{equation}
			\boxed{
				\tag{3.11}
				\begin{split}
					G = \frac{E}{2\left(1+\eta\right)}
				\end{split}
			}
		\end{equation}

		\begin{equation}
			\boxed{
				\tag{3.12a}
				\begin{split}
					\varepsilon_x &= \frac{1}{E}\left(\sigma_x - \nu \sigma_y\right)
					\\
					\varepsilon_y &= \frac{1}{E}\left(\sigma_y - \nu \sigma_x\right)
					\\
			\gamma_{xy} &= \frac{1}{G}\tau_{xy}
				\end{split}
			}
		\end{equation}

		\begin{equation}
			\boxed{
				\tag{3.12b}
				\begin{split}
					\sigma_x &= \frac{E}{1 - \nu^2}(\epsilon_x + \nu \epsilon_y) \\
					\sigma_y &= \frac{E}{1 -\nu^2}(\epsilon_y - \nu \epsilon_x) \\
					\tau_{xy} &= G \gamma_{xy}
				\end{split}
			}
		\end{equation}

		\begin{equation}
			\boxed{
				\tag{3.13}
				\begin{split}
					\epsilon_1 = \frac{1}{E}(\sigma_1 -\nu\sigma_2), \quad \epsilon_2 = \frac{1}{E}(\sigma_2 -\nu\sigma_1)
				\end{split}
			}
		\end{equation}

		\begin{equation}
			\boxed{
				\tag{3.14}
				\begin{split}
					\varepsilon_x &= \frac{1}{E}\left[\sigma_x -\nu   \left(\sigma_y + \sigma_z\right)\right] + \alpha_T\Delta T
					\\
					\varepsilon_y &= \frac{1}{E}\left[\sigma_y -\nu   \left(\sigma_z + \sigma_x\right)\right] + \alpha_T\Delta T\\
					\varepsilon_z &= \frac{1}{E}\left[\sigma_z -\nu   \left(\sigma_x + \sigma_y\right)\right] + \alpha_T\Delta T\\
					\gamma_{xy} &= \frac{1}{G}\tau_{xy}, \quad \gamma_{xz} = \frac{1}{G}\tau_{xz}, \quad \gamma_{yz} = \frac{1}{G}\tau_{yz}
				\end{split}
			}
		\end{equation}

		\subsection{Festigkeitshypothesen}

		\begin{equation}
			\boxed{
				\tag{3.15}
				\begin{split}
					\sigma_V \leq \sigma_{zul}
				\end{split}
			}
		\end{equation}

		\begin{equation}
			\boxed{
				\tag{3.16}
				\begin{split}
					\sigma_V= \sigma_1
				\end{split}
			}
		\end{equation}

		\begin{equation}
			\boxed{
				\tag{3.17}
				\begin{split}
					\sigma_V= \sqrt{ \left( \sigma_x - \sigma_y \right)^2 + 4\tau^2_{xy} }
				\end{split}
			}
		\end{equation}

		\begin{equation}
			\boxed{
				\tag{3.18}
				\begin{split}
					\sigma_V= \sqrt{\sigma_x^2 + \sigma_y^2 -\sigma_x\sigma_y + 3\tau^2_{xy} }
				\end{split}
			}
		\end{equation}

		\subsection{Zusammenfassung}

		\section{Balkenbiegung}
		\subsection{Einführung}

		\begin{equation}
			\boxed{
				\tag{4.1}
				\begin{split}
					\sigma(z) = c z
				\end{split}
			}
		\end{equation}

		\begin{equation}
			\boxed{
				\tag{4.2}
				\begin{split}
					M = \int{z\sigma \; \text{d}A}
				\end{split}
			}
		\end{equation}

				\begin{equation}
			\boxed{
				\tag{4.3}
				\begin{split}
					I = \int z^2 \td A
				\end{split}
			}
		\end{equation}

						\begin{equation}
			\boxed{
				\tag{4.4}
				\begin{split}
					\sigma = \frac{M}{I} z
				\end{split}
			}
		\end{equation}

		\subsection{Flächenträgheitsmomente}
		\subsubsection{Definition}

		Das statische Moment ist quasi Fläche $\times$ Hebelarm bezogen auf den Schwerpunkt der Fläche:
		\begin{equation}
			\boxed{
				\tag{4.5}
				\begin{split}
					S_y
					=
					\int z \text{d} A, \quad
					S_z
					=
					\int y \text{d} A
				\end{split}
			}
		\end{equation}

		\begin{equation}
			\boxed{
				\tag{4.6a}
				\begin{split}
					I_y
					=
					\int
					z^2\text{d}A, \quad
					I_z
					=
					\int
					y^2\text{d}A
				\end{split}
			}
		\end{equation}
		
		\begin{equation}
			\boxed{
				\tag{4.6b}
				\begin{split}
					I_{yz} = I_{zy} = -\!\int\! y z\, \text{d}A
				\end{split}
			}
		\end{equation}
		
		\begin{equation}
			\boxed{
				\tag{4.6c}
				\begin{split}
					I_p = \int r^2 \,\text{d}A = \int\! \left( z^2 + y^2  \right)\,\text{d}A = I_y + I_z
				\end{split}
			}
		\end{equation}

		\begin{equation}
			\boxed{
				\tag{4.7}
				\begin{split}
					i = seltsame Wurzel; 
					\text{da bin ich jetzt zu faul}
				\end{split}
			}
		\end{equation}

%		\begin{equation}
%			\boxed{
%				\tag{4.8a}
%				\begin{split}
%					I_y
%					=
%					\int
%					z^2\text{d}A
%					=
%					\int\limits_{-h/2}^{+h/2}z^2(b\, \text{d}z)
%					=
%					\left[\frac{b z^3}{3}\right]_{-h/2}^{+h/2} % fucking limits, dont use them !!!
%					=
%					\frac{bh^3}{12}
%				\end{split}
%			}
%		\end{equation}
%
%		Für das Rechteck in der Graphik:
%		\begin{equation}
%			\boxed{
%				\tag{4.8b}
%				\begin{split}
%					I_z = \frac{h b^3}{12}
%				\end{split}
%			}
%		\end{equation}
%		
%		In dem Beispiel wegen der Symmetrieachse:
%		\begin{equation}
%			\boxed{
%				\tag{4.8c}
%				\begin{split}
%					I_{yz} = 0
%				\end{split}
%			}
%		\end{equation}
%
%		\begin{equation}
%			\boxed{
%				\tag{4.8d}
%				\begin{split}
%					\text{da bin ich jetzt zu faul}
%				\end{split}
%			}
%		\end{equation}
%
%		\begin{equation}
%			\boxed{
%				\tag{4.8e}
%				\begin{split}
%					\text{da bin ich jetzt zu faul}
%				\end{split}
%			}
%		\end{equation}
%
%		\begin{equation}
%			\boxed{
%				\tag{4.9}
%				\begin{split}
%					\text{da bin ich jetzt zu faul}
%				\end{split}
%			}
%		\end{equation}
%
%		\begin{equation}
%			\boxed{
%				\tag{4.10a}
%				\begin{split}
%					\text{da bin ich jetzt zu faul}
%				\end{split}
%			}
%		\end{equation}
%
%		\begin{equation}
%			\boxed{
%				\tag{4.10b}
%				\begin{split}
%					\text{da bin ich jetzt zu faul}
%				\end{split}
%			}
%		\end{equation}
%
%		\begin{equation}
%			\boxed{
%				\tag{4.10c}
%				\begin{split}
%					\text{da bin ich jetzt zu faul}
%				\end{split}
%			}
%		\end{equation}
%
%		\begin{equation}
%			\boxed{
%				\tag{4.11}
%				\begin{split}
%					\text{da bin ich jetzt zu faul}
%				\end{split}
%			}
%		\end{equation}
%
%		\begin{equation}
%			\boxed{
%				\tag{4.12}
%				\begin{split}
%					\text{da bin ich jetzt zu faul}
%				\end{split}
%			}
%		\end{equation}

		\subsubsection{Parallelverschiebung der Bezugsachsen}

		\begin{equation}
			\boxed{
				\tag{4.13}
				\begin{split}
					I_{\bar{y}} &= I_y + \bar{z}^2_s A\\
					I_{\bar{z}} &= I_z + \bar{y}^2_s A\\
					I_{\bar{y} \bar{z}} &= I_{yz} - \bar{y}_{s} \bar{z}_{s} A
				\end{split}
			}
		\end{equation}

		\subsubsection{Drehung des Bezugssystems, Hauptträgheitsmomente}

		\begin{equation}
			\boxed{
				\tag{4.14}
				\begin{split}
					I_{\eta} &= \frac{1}{2}\left( I_y + I_z \right) &+ \frac{1}{2}\left( I_y - I_z \right)\cos{2\varphi} + I_{yz} \sin{2\varphi}\\
					I_{\zeta} &= \frac{1}{2}\left( I_y - I_z \right)&-\frac{1}{2}(I_y-I_z)\cos{2\varphi} -I_{yz}\sin{2\varphi}\\
					I_{\eta \zeta} &=  &-\frac{1}{2}\left( I_y - I_z \right)\sin{2\varphi} + I_{yz}\cos{2\varphi}
				\end{split}
			}
		\end{equation}

		\begin{equation}
			\boxed{
				\tag{4.15}
				\begin{split}
					I_{\eta} + I_{\zeta} = I_y + I_z = I_p
				\end{split}
			}
		\end{equation}

		\begin{equation}
			\boxed{
				\tag{4.16}
				\begin{split}
					\tan{2\varphi^*} = \frac{2 I_{yz}}{I_y - I_z}
				\end{split}
			}
		\end{equation}

		\begin{equation}
			\boxed{
				\tag{4.17}
				\begin{split}
					I_{1,2} = \frac{I_y + I_z}{2}
					\pm
					\sqrt{\left(\frac{I_y - I_z}{2}\right)^2 + I_{yz}^2}		
				\end{split}
			}
		\end{equation}
		
		\subsection{Grundgleichungen der geraden Biegung}

		\begin{equation}
			\boxed{
				\tag{4.18}
				\begin{split}
					\frac{\text{d}Q}{\text{d}x} = -q,\quad \frac{\text{d}M}{\text{d}x} = Q
				\end{split}
			}
		\end{equation}

		\begin{equation}
			\boxed{
				\tag{4.19a}
				\begin{split}
					M = \int z \sigma \td A
				\end{split}
			}
		\end{equation}


		\begin{equation}
			\boxed{
				\tag{4.19b}
				\begin{split}
					Q = \int \tau \td A
				\end{split}
			}
		\end{equation}

		\begin{equation}
			\boxed{
				\tag{4.19c}
				\begin{split}
					N = \int \sigma \td A
				\end{split}
			}
		\end{equation}

		\begin{equation}
			\boxed{
				\tag{4.20}
				\begin{split}
					\epsilon = \frac{\partial u}{\partial x}
				\end{split}
			}
		\end{equation}

		\begin{equation}
			\boxed{
				\tag{4.21}
				\begin{split}
					\sigma= E\,\epsilon, \quad \tau = G\,\gamma
				\end{split}
			}
		\end{equation}

		\begin{equation}
			\boxed{
				\tag{4.22a}
				\begin{split}
					\omega = \omega(x)
				\end{split}
			}
		\end{equation}

		\begin{equation}
			\boxed{
				\tag{4.22b}
				\begin{split}
					u(x,z) = \psi(x)z
				\end{split}
			}
		\end{equation}
		
		\begin{equation}
			\boxed{
				\tag{4.23a}
				\begin{split}
					\sigma = E\frac{\partial u}{\partial x} = E \psi' z
				\end{split}
			}
		\end{equation}

		\begin{equation}
			\boxed{
				\tag{4.23b}
				\begin{split}
					\tau = G\left( \frac{\partial \omega}{\partial x} + \frac{\partial u}{\partial z} \right) = G (\omega' + \psi)
				\end{split}
			}
		\end{equation}

		\begin{equation}
			\boxed{
				\tag{4.24}
				\begin{split}
					M = E I \psi'
				\end{split}
			}
		\end{equation}

		\begin{equation}
			\boxed{
				\tag{4.25}
				\begin{split}
					Q = \varkappa GA(\omega' +\psi)
				\end{split}
			}
		\end{equation}
		
		\subsection{Normalspannungen}

		\begin{equation}
			\boxed{
				\tag{4.26}
				\begin{split}
					\sigma
					=
					\frac{M}{I}
					z
				\end{split}
			}
		\end{equation}
		
		\begin{equation}
			\boxed{
				\tag{4.27}
				\begin{split}
					W
					=
					\frac{I}{\abs{z}_{\text{max}}}
				\end{split}
			}
		\end{equation}
		Aber hier mit subscript, also $\displaystyle W_{\text{Achse}} = \frac{I_{\text{Achse}}}{\abs{\text{andere Achse}}_{\text{max}}}$
		\begin{equation}
			\boxed{
				\tag{4.28}
				\begin{split}
					\sigma_{\text{max}}
					=
					\frac{\abs{M}}{W}
				\end{split}
			}
		\end{equation}
		
		\subsection{Biegelinie}
		\subsubsection{Differentialgleichung der Biegelinie}

		\begin{equation}
			\boxed{
				\tag{4.29}
				\begin{split}
					\omega' + \psi = 0
				\end{split}
			}
		\end{equation}

		\begin{equation}
			\boxed{
				\tag{4.30}
				\begin{split}
					Q' = -q,\quad M' = Q, \quad \psi' = \frac{M}{E I}, \quad \omega' = -\psi
				\end{split}
			}
		\end{equation}

		\begin{equation}
			\boxed{
				\tag{4.31}
				\begin{split}
					\omega'' = -\frac{M}{E I}
				\end{split}
			}
		\end{equation}

		\begin{equation}
			\boxed{
				\tag{4.32a}
				\begin{split}
					\varkappa_B = \frac{\omega''}{(1 + \omega'^2)^{\nicefrac{3}{2}} }
				\end{split}
			}
		\end{equation}

		\begin{equation}
			\boxed{
				\tag{4.32b}
				\begin{split}
					\varkappa_B\approx \omega''
				\end{split}
			}
		\end{equation}
		
		\begin{equation}
			\boxed{
				\tag{4.33}
				\begin{split}
					Q = -(E I \omega'')'
				\end{split}
			}
		\end{equation}

		\begin{equation}
			\boxed{
				\tag{4.34a}
				\begin{split}
					(E I \omega'')'' = q
				\end{split}
			}
		\end{equation}

		\begin{equation}
			\boxed{
				\tag{4.34b}
				\begin{split}
					EI \omega^{I V} = q
				\end{split}
			}
		\end{equation}
		
		\subsubsection{Einfeldbalken} 
		\subsubsection{Balken mit mehreren Feldern}
		\subsubsection{Superposition}
		\subsection{Einfluss des Schubes}
		\subsubsection{Schubspannungen}

		\begin{equation}
			\boxed{
				\tag{4.35}
				\begin{split}
					\frac{\partial \sigma}{\partial x} = \frac{Q}{I} \zeta
				\end{split}
			}
		\end{equation}

		\begin{equation}
			\boxed{
				\tag{4.36}
				\begin{split}
					S(z) = \int\limits_{A^*} \zeta \td A
				\end{split}
			}
		\end{equation}

		\begin{equation}
			\boxed{
				\tag{4.37}
				\begin{split}
					\underbrace{\tau(z)}_{\nicefrac{N}{mm^2}} = \frac{\overbrace{Q}^{[N]}\overbrace{S(z)}^{mm^3}}{\underbrace{I}_{mm^4}\underbrace{b(z)}_{mm}}
				\end{split}
			}
		\end{equation}
		
		\subsubsection{Durchbiegung infolge Schub}

		\begin{equation}
			\boxed{
				\tag{4.40}
				\begin{split}
					\omega' + \psi = \frac{Q}{GA_S}
				\end{split}
			}
		\end{equation}

		\begin{equation}
			\boxed{
				\tag{4.41}
				\begin{split}
					\omega_s' = \frac{Q}{GA_S}
				\end{split}
			}
		\end{equation}

		\begin{equation}
			\boxed{
				\tag{4.42}
				\begin{split}
					\omega' = \omega_B' + \omega_S'
				\end{split}
			}
		\end{equation}

		\begin{equation}
			\boxed{
				\tag{4.43}
				\begin{split}
					\omega = \omega_B + \omega_S
				\end{split}
			}
		\end{equation}
		
		\begin{equation}
			\boxed{
				\tag{4.44}
				\begin{split}
					\omega_S= \frac{F}{GA_S}x
				\end{split}
			}
		\end{equation}

		\subsection{Schiefe Biegung}

		\begin{equation}
			\boxed{
				\tag{4.45}
				\begin{split}
					\sigma = \frac{M_y}{I_y}z - \frac{M_z}{I_z}y
				\end{split}
			}
		\end{equation}
		
		\begin{equation}
			\boxed{
				\tag{4.46}
				\begin{split}
					\omega'' = \frac{M_y}{EI_y} , \quad \nu '' = \frac{M_z}{EI_z}
				\end{split}
			}
		\end{equation}

		\begin{equation}
			\boxed{
				\tag{4.47}
				\begin{split}
					\frac{\td Q_z}{\td x} = -q_z, \quad \frac{\td Q_y}{\td x} &= -q_y\\
					\frac{\td M_y}{\td x} = Q_z, \quad \frac{\td M_z}{\td x} &= -Q_y
				\end{split}
			}
		\end{equation}

		\begin{equation}
			\boxed{
				\tag{4.48}
				\begin{split}
					\epsilon = -\left(\omega'' z + \nu'' y\right)
				\end{split}
			}
		\end{equation}
		
		\begin{equation}
			\boxed{
				\tag{4.49}
				\begin{split}
					\sigma = -E\left(\omega'' z + \nu'' y\right)
				\end{split}
			}
		\end{equation}
		
		\begin{equation}
			\boxed{
				\tag{4.50}
				\begin{split}
					M_y = \int z\sigma\td A , \quad M_z = - \int y \sigma \td A
				\end{split}
			}
		\end{equation}
		
		\begin{equation}
			\boxed{
				\tag{4.51}
				\begin{split}
					\text{da bin ich jetzt zu faul}
				\end{split}
			}
		\end{equation}
		
		\begin{equation}
			\boxed{
				\tag{4.52}
				\begin{split}
					\text{da bin ich jetzt zu faul}
				\end{split}
			}
		\end{equation}

		\begin{equation}
			\boxed{
				\tag{4.53a}
				\begin{split}
					\text{da bin ich jetzt zu faul}
				\end{split}
			}
		\end{equation}

		\begin{equation}
			\boxed{
				\tag{4.53b}
				\begin{split}
					\text{da bin ich jetzt zu faul}
				\end{split}
			}
		\end{equation}

		\subsection{Biegung und Zug/Druck}

		\begin{equation}
			\boxed{
				\tag{4.54a}
				\begin{split}
					\text{da bin ich jetzt zu faul}
				\end{split}
			}
		\end{equation}
		
		\begin{equation}
			\boxed{
				\tag{4.54b}
				\begin{split}
					\text{da bin ich jetzt zu faul}
				\end{split}
			}
		\end{equation}
		
		\subsection{Kern des Querschnitts}

		\begin{equation}
			\boxed{
				\tag{4.55}
				\begin{split}
					\text{da bin ich jetzt zu faul}
				\end{split}
			}
		\end{equation}
		
		\begin{equation}
			\boxed{
				\tag{4.56}
				\begin{split}
					\text{da bin ich jetzt zu faul}
				\end{split}
			}
		\end{equation}

		\begin{equation}
			\boxed{
				\tag{4.57}
				\begin{split}
					\text{da bin ich jetzt zu faul}
				\end{split}
			}
		\end{equation}
		\subsection{Temperaturbelastung}

		\begin{equation}
			\boxed{
				\tag{4.58}
				\begin{split}
					\text{da bin ich jetzt zu faul}
				\end{split}
			}
		\end{equation}
		
		\begin{equation}
			\boxed{
				\tag{4.59}
				\begin{split}
					\text{da bin ich jetzt zu faul}
				\end{split}
			}
		\end{equation}

		\begin{equation}
			\boxed{
				\tag{4.60}
				\begin{split}
					\text{da bin ich jetzt zu faul}
				\end{split}
			}
		\end{equation}

		\begin{equation}
			\boxed{
				\tag{4.61}
				\begin{split}
					\text{da bin ich jetzt zu faul}
				\end{split}
			}
		\end{equation}

		\begin{equation}
			\boxed{
				\tag{4.62}
				\begin{split}
					\text{da bin ich jetzt zu faul}
				\end{split}
			}
		\end{equation}
		
		\begin{equation}
			\boxed{
				\tag{4.63}
				\begin{split}
					\text{da bin ich jetzt zu faul}
				\end{split}
			}
		\end{equation}

		\begin{equation}
			\boxed{
				\tag{4.64}
				\begin{split}
					\text{da bin ich jetzt zu faul}
				\end{split}
			}
		\end{equation}

		\begin{equation}
			\boxed{
				\tag{4.65}
				\begin{split}
					\text{da bin ich jetzt zu faul}
				\end{split}
			}
		\end{equation}

		\subsection{Zusammenfassung}
	
		\section{Torsion}
		\subsection{Einführung}
		\subsection{Die kreiszylindrische Welle}

		\begin{equation}
			\boxed{
				\tag{5.1}
				\begin{split}
					r\td \theta = \gamma \td x \rightarrow \gamma = r\frac{\td \theta}{\td x}
				\end{split}
			}
		\end{equation}
		Man nennt die Verdrehung pro Längeneinheit $ \td\theta = \td x $ manchmal auch Verwindung $\varkappa_T$.
		
		\begin{equation}
			\boxed{
				\tag{5.2}
				\begin{split}
					\tau = G r \frac{\td \theta}{\td x} = G r \theta'
				\end{split}
			}
		\end{equation}

		\begin{equation}
			\boxed{
				\tag{5.3}
				\begin{split}
					M_T = \int r\theta \td A
				\end{split}
			}
		\end{equation}

		\begin{equation}
			\boxed{
				\tag{5.4}
				\begin{split}
					M_T = G\theta' \displaystyle\int r^2 \td A = G \theta' I_p
				\end{split}
			}
		\end{equation}
		
		\[\int\]
		$\int $
		
		\begin{equation}
			\boxed{
				\tag{5.5}
				\begin{split}
					GI_T \theta' = M_T
				\end{split}
			}
		\end{equation}
		Die Größe $GI_T$ heißt Torsionssteifigkeit.

		\begin{equation}
			\boxed{
				\tag{5.6}
				\begin{split}
					M_T = M_x
				\end{split}
			}
		\end{equation}

		\begin{equation}
			\boxed{
				\tag{5.7}
				\begin{split}
					\theta_l = \frac{M_T l}{GI_T}
				\end{split}
			}
		\end{equation}

		\begin{equation}
			\boxed{
				\tag{5.8}
				\begin{split}
					\tau = \frac{M_T}{I_T}r
				\end{split}
			}
		\end{equation}
		
		\pagebreak[0]
		Der Größtwert tritt am Rand $ r = R $ auf: $ \tau_{\text{max}} = \left( M_T / I_T \right) R $.
		Um
die Analogie zur Biegung herzustellen, führen wir ein \textit{Torsionswiderstandsmoment} $ W_T $ ein:
		\nopagebreak
		\begin{equation}
			\boxed{
				\tag{5.9}
				\begin{split}
					\tau_{\text{max}} = \frac{M_T}{W_T}
				\end{split}
			}
		\end{equation}

		\begin{equation}
			\boxed{
				\tag{5.10}
				\begin{split}
					I_t = I_P = \frac{\pi}{2} R^4, \quad W_T = \frac{\pi}{2}R^3
				\end{split}
			}
		\end{equation}

		\begin{equation}
			\boxed{
				\tag{5.11}
				\begin{split}
					I_T = \frac{\pi}{2} \left( R^4_a - R^4_i \right), \quad W_T = \frac{\pi}{2}\frac{R^4_a -R_i^4}{R_a}
				\end{split}
			}
		\end{equation}

		\begin{equation}
			\boxed{
				\tag{5.12}
				\begin{split}
					I_T \approx 2 \pi R^3_m t \quad W_T \approx 2 \pi R^2_m t
				\end{split}
			}
		\end{equation}

		\begin{equation}
			\boxed{
				\tag{5.13}
				\begin{split}
					\frac{\td M_T}{\td x} = M'_T = -m_T
				\end{split}
			}
		\end{equation}

		\begin{equation}
			\boxed{
				\tag{5.14}
				\begin{split}
					\left( GI_T \theta' \right)' = -m_T
				\end{split}
			}
		\end{equation}

		\subsection{Dünnwandige geschlossene Profile}

		\begin{equation}
			\boxed{
				\tag{5.15}
				\begin{split}
					T = \tau t
				\end{split}
			}
		\end{equation}

		\begin{equation}
			\boxed{
				\tag{5.16}
				\begin{split}
					T = \tau t = \text{const}
				\end{split}
			}
		\end{equation}

		\begin{equation}
			\boxed{
				\tag{5.17}
				\begin{split}
					M_T = \oint \td M_T = T \oint r \perp\td s
				\end{split}
			}
		\end{equation}

		\begin{equation}
			\boxed{
				\tag{5.18}
				\begin{split}
					\oint r \perp \td s = 2 A_m
				\end{split}
			}
		\end{equation}

		\begin{equation}
			\boxed{
				\tag{5.19}
				\begin{split}
					M_T = 2A_m T
				\end{split}
			}
		\end{equation}

		\begin{equation}
			\boxed{
				\tag{5.20}
				\begin{split}
					\tau = \frac{T}{t} = \frac{M_T}{2 A_m t}
				\end{split}
			}
		\end{equation}

		\begin{equation}
			\boxed{
				\tag{5.21}
				\begin{split}
					\tau_{\text{max}} = \frac{M_T}{W_T} \quad \text{mit} \quad W_T = 2 A_m t_{\text{min}}
				\end{split}
			}
		\end{equation}

		\begin{equation}
			\boxed{
				\tag{5.22}
				\begin{split}
					\td \nu = r \perp \td \theta
				\end{split}
			}
		\end{equation}

		\begin{equation}
			\boxed{
				\tag{5.23}
				\begin{split}
					\frac{T}{G t} = r \perp \theta' + \frac{\partial u}{\partial s}
				\end{split}
			}
		\end{equation}

		\begin{equation}
			\boxed{
				\tag{5.24}
				\begin{split}
					\theta' = \frac{M_T}{GI_T}
				\end{split}
			}
		\end{equation}

		\begin{equation}
			\boxed{
				\tag{5.25}
				\begin{split}
					I_T = \frac{\left( 2 A_m\right)^2}{\oint\frac{\td s}{t}}
				\end{split}
			}
		\end{equation}

		\begin{equation}
			\boxed{
				\tag{5.26}
				\begin{split}
					\text{da bin ich jetzt zu faul}
				\end{split}
			}
		\end{equation}

		\begin{equation}
			\boxed{
				\tag{5.27}
				\begin{split}
					\text{da bin ich jetzt zu faul}
				\end{split}
			}
		\end{equation}
		
		\begin{equation}
			\boxed{
				\tag{5.28}
				\begin{split}
					\text{da bin ich jetzt zu faul}
				\end{split}
			}
		\end{equation}

		\subsection{Dünnwandige offene Profile}

		\begin{equation}
			\boxed{
				\tag{5.29}
				\begin{split}
					\text{da bin ich jetzt zu faul}
				\end{split}
			}
		\end{equation}

		\begin{equation}
			\boxed{
				\tag{5.30}
				\begin{split}
					\text{da bin ich jetzt zu faul}
				\end{split}
			}
		\end{equation}

		\begin{equation}
			\boxed{
				\tag{5.31}
				\begin{split}
					\text{da bin ich jetzt zu faul}
				\end{split}
			}
		\end{equation}

		\begin{equation}
			\boxed{
				\tag{5.32}
				\begin{split}
					\text{da bin ich jetzt zu faul}
				\end{split}
			}
		\end{equation}

		\begin{equation}
			\boxed{
				\tag{5.33}
				\begin{split}
					\text{da bin ich jetzt zu faul}
				\end{split}
			}
		\end{equation}

		\begin{equation}
			\boxed{
				\tag{5.34}
				\begin{split}
					\text{da bin ich jetzt zu faul}
				\end{split}
			}
		\end{equation}
		
		\subsection{Zusammenfassung}	

		\section{Der Arbeitsbegriff in der Elastostatik}
		\subsection{Einleitung}
		\subsection{Arbeitssatz und Formänderungsenergie}
		\subsection{Das Prinzip der virtuellen Kräfte}
		\subsection{Einflusszahlen und Vertauschungssätze}
		\subsection{Anwendung des Arbeitssatzes auf statisch unbestimmte Systeme}
		\subsection{Zusammenfassung}

		\section{Knickung}
		\subsection{Verzweigung einer Gleichgewichtslage}
		\subsection{Der Euler-Stab}
		\subsection{Zusammenfassung}
	
		\section{Verbundquerschnitte}
		\subsection{Einleitung}
		\subsection{Zug und Druck in Stäben}
		\subsection{Reine Biegung}
		\subsection{Biegung und Zug/Druck}
		\subsection{Zusammenfassung}
\end{document}














































