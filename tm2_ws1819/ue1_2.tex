Gegeben: $F, a, EA$

\begin{itemize}
\item[a.)]
  An welcher Stelle $x$ muss die Kraft $F$ angreifen, damit der Balken in horizontaler Lage verbleibt?

  \item[b.)]
    Wie groß sond dann die Spannungen in den Pfosten?

  \item[c.)]
    Welche Schrägstellung des Balkens tritt für $x = \nicefrac{2}{3}a$ auf?  
\end{itemize}

Sei A links oben, (x=0)
\begin{equation*}
  \begin{split}
    \sum \stackrel{\curvearrowleft}{A} \colon  a \cdot F_{b2}  + 2a \cdot F_{b3} - x \cdot F\\
    x = ?...
  \end{split}
\end{equation*}


GH 1.18 anwenden für jeden Stab separat, dann die drei $\Delta \ell$ als gleich annehmen (keine änderung in der winkel-lage des balkens)\\

dann nach den relativen stabkräften auflösen, also summe vertikaler kräfte mit nur einem $s_i$

dann summe momente $\rightarrow$ x
