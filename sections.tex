    \abovedisplayskip = 1pt plus 10pt minus 0pt
    \belowdisplayskip = 1pt plus 10pt minus 0pt
    \abovedisplayshortskip = 0pt plus 10pt minus 0pt
    \belowdisplayshortskip = 0pt plus 10pt minus 0pt
    %\pagestyle{fancy}%
    \begin{dmath*}[frame]
        1 \text{Newton} \coloneqq \frac{kg \cdot \text{meter}}{\text{sekunde}^2}
    \end{dmath*}

    \hrulefill
    
    Spannung:
    \begin{dmath*}[frame]
        \underbrace{\sigma}_{\text{Spannung}\left[\frac{N}{mm^2}\right]}
        =
        \frac{\overbrace{F}^{\text{Kraft}\left[N\right]}}{\underbrace{A}_{\text{Fläche}\left[mm^2\right]}}
    \end{dmath*}

    \hrulefill
    \begin{dmath*}[frame]
        \underbrace{F_G}_{\text{Gewichtskraft} [N = \frac{kg m}{s^2}]} = \underbrace{m}_{\text{Masse}\left[kg\right]} \cdot \underbrace{g}_{\text{Fallbeschleunigung} \left[\frac{m}{s^2}\right]}
        = \underbrace{V}_{\text{Volumen}\left[m^3\right]} \cdot \underbrace{\rho}_{\text{Dichte}\left[\frac{kg}{m^3}\right]} \cdot g
    \end{dmath*}
    
    \hrulefill
    \begin{dmath*}[frame]
        \underbrace{\Delta \ell}_{\text{Verlängerung}\left[m\right]} = \underbrace{\ell}_{\text{belastete Länge} \left[m\right]} - \underbrace{\ell_0}_{\text{Urprungslänge} \left[m\right]}
    \end{dmath*}
    Die Verlängerung $\Delta \ell$ ist $>0$ wenn Das Teil länger wird, daran gezogen wird.\\
    Die Verlängerung $\Delta \ell$ ist $<0$ wenn Das Teil kürzer wird, daran gedrückt wird.

    \hrulefill
    
    $\varepsilon$ ist die Dehnung als relative Angabe, also in \%.
    \begin{dmath*}[frame]
        \underbrace{\varepsilon}_{\text{Dehnung} \left[Einheitslos \; \widehat{=}\; 1\right]} 
        =
        \frac{\overbrace{\Delta \ell}^{\text{Verlängerung}\left[m\right]}}{\underbrace{\ell_0}_{\text{Ursprungslänge}\left[m\right]}}
        = \frac{\ell - \ell_0}{\ell_0}
    \end{dmath*}

    \hrulefill
    
    Querdehnung, Änderung der Dicke durch Belastung normal dazu.
    \begin{dmath*}[frame]
        \underbrace{\varepsilon_q}_{\text{Querdehnung} \left[1\right]} 
        =
        \frac{\overbrace{\Delta d}^{\text{Dickenänderung}\left[m\right]}}{\underbrace{d_0}_{\text{Ursprüngliche Dicke} \left[m\right]}}
        = \frac{d - d_0}{d_0}
    \end{dmath*}    

    \hrulefill
    \begin{dmath*}[frame]
        \underbrace{m}_{\text{Poisson-Zahl}[1]} = \frac{\overbrace{\varepsilon}^{\text{Dehnung} [1]}}{\underbrace{\varepsilon_q}_{\text{Querdehnung}[1]}}
    \end{dmath*}
    Auch als Kehrwert genutzt:
    \begin{dmath*}[frame]
        \underbrace{\mu}_{\text{Querzahl oder Querkontraktionszahl}[1]} = \frac{1}{\underbrace{m}_{\text{Poisson-Zahl}[1]}}
    \end{dmath*}

    \hrulefill
        
    Hookesches Gesetz:
    \begin{dmath*}[frame]
        \underbrace{E}_{\text{Elastizitätsmodul}\left[\frac{N}{mm^2}\right]}
        =
        \frac{\overbrace{\sigma}^{\text{Spannung}\left[\frac{N}{mm^2}\right]}}
        {\underbrace{\varepsilon}_{\text{Dehnung}\left[1\right]}}
    \end{dmath*}
    Umgestellt nach Sigma, übliche Form:
    \begin{dmath*}[frame]
        \sigma
        =
        \varepsilon E
        =
        \frac{\Delta \ell}{\ell_0}E
    \end{dmath*}
    \hrulefill

    Wärmespannung:
    \begin{dmath*}[frame]
        \underbrace{\Delta \ell}_{\left[ mm \right]}
        =
        \underbrace{\ell_0}_{\text{Ursprungslänge}\left[mm\right]} \cdot
        \underbrace{\alpha_{\ell}}_{\text{Längenausdehnungskoeffizient}\left[\frac{1}{K}\right]} \cdot
        \underbrace{\Delta T}_{\text{Temperaturunterschied}\left[K\right]} \cdot
    \end{dmath*}
    \hrulefill

    Abscherspannung: (Ananlog Spannung $\sigma$, auch Abscherspannung genannt)
    \begin{dmath*}[frame]
        \underbrace{\tau}_{\text{Abscherspannung}\left[\frac{N}{mm^2}\right]}
        =
        \frac{\overbrace{F_q}^{\text{Querkraft}\left[N\right]}}
             {\underbrace{A}_{\text{Querschnittsfläche}\left[mm^2\right]}}
    \end{dmath*}
    
    \hrulefill
