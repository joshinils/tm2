\documentclass[11pt]{article}

\usepackage{amsmath,amssymb, a4, verbatim}
\usepackage[german]{babel}
%\usepackage[latin1]{inputenc}
\usepackage[utf8]{inputenc} % üöäß
\usepackage{listings} % für inline codelistings
\lstset{%
    basicstyle=\ttfamily,    % the size of the fonts
    columns=fixed,            % anything else is horrifying
    showspaces=false,        % show spaces using underscores?
    showstringspaces=false,    % underline spaces within strings?
    showtabs=false,            % show tabs within strings?
    xleftmargin=1.5em,        % left margin space
}
\lstdefinestyle{inline}{basicstyle=\ttfamily}
\newcommand{\listline}[1]{\lstinline[style=inline]!#1!}

\usepackage{caption}
\newcommand{\tinycaption}[1]{\captionsetup{labelformat=empty}\caption{#1}}

%\usepackage{color}
%\usepackage{epsfig} % eps
\usepackage{graphicx} % eps
%\usepackage[shortcuts]{extdash}
%\usepackage{dsfont}
%\usepackage{epstopdf} % eps
%\usepackage[pdf]{pstricks} % eps
%\usepackage{auto-pst-pdf}
\usepackage{mathtools}
\usepackage{dsfont} % $ \mathds{1} $
\usepackage{icomma}
\usepackage{tikz}
%\usepackage{pgfplots}

%\pgfplotsset{compat=1.8}
\usepackage[bottom]{footmisc} % put footnotes at the bottom of page
\usepackage{nicefrac} % für brüche die aussehen wie prozentzeichen
% \usepackage{ps2pdf}
\usetikzlibrary{automata,positioning}

\usepackage{algorithmicx}
\usepackage{algpseudocode}
\usepackage{algorithm}

\usepackage{multicol}
\usepackage{wrapfig} % make stuff float
\usepackage{placeins} % stop stuff from floating
\usepackage{seqsplit} % very long numbers
\usepackage{framed} % begin{framed}

%  Headings and Footings :
\usepackage{fancyhdr}
\headheight15pt
\lhead{Technische Mechanik II, \ueberschrift}

\chead{}
\rhead{\thepage}
\renewcommand{\headrulewidth}{.4pt}

\lfoot{\today}
\cfoot{}
\rfoot{Joshua}
\renewcommand{\footrulewidth}{.4pt}

%----------------------------------------------------------------

\textwidth16.5cm
\oddsidemargin0.cm
\evensidemargin0.cm

\parindent0cm

\newcommand{\R}{ {\mathbb R} }
\newcommand{\C}{ {\mathbb C} }
\newcommand{\1}{ {\mathds{1}} }
\newcommand{\abs}[1]{\lvert#1\rvert}
\newcommand{\norm}[1]{\left\lVert#1\right\rVert}
\newcommand{\xt}{\tilde{x}}
\newcommand{\dotleq}{\dot{\leq}}
\newcommand{\m}{\hphantom{-} }

\newcommand{\dashfill}[1]{\vspace{11pt}\def\dashfill{\cleaders\hbox{#1}\hfill}\hbox to \hsize{\dashfill\hfil}\vspace{11pt}}
\newcommand{\scdot}{\!\cdot\!}


\newcommand{\sig}{\text{signum}}
\newcommand{\rot}{}

% ------------------  edit Ueberschrift ---------------------
\newcommand{\ueberschrift}{TM II Notizen}


\usepackage{mathtools}
\usepackage{ragged2e}
\newlength\ubwidth
\newlength\obwidth
\newcommand\underbraceWrap[3][0pt]
{
  \settowidth\ubwidth{$#1$}
  \underbrace{#2}_
  {
    \parbox
      {
        \maxof{\ubwidth}{\numexpr#1}
      }
      {
        \scriptsize\Centering#3
      }
  }
}
\newcommand\overbraceWrap[3][0pt]
{
  \settowidth\obwidth{$#1$}
  \overbrace{#2}^
    {
      \parbox
        {
          \maxof{\obwidth}{\numexpr#1}
        }
        {
          \scriptsize\Centering#3
        }
    }
}

% -----------------------------------------------------------
\begin{document}
    \abovedisplayskip = 5pt plus 40pt minus 0pt
    \belowdisplayskip = 5pt plus 40pt minus 0pt
    \abovedisplayshortskip = 0pt plus 40pt minus 0pt
    \belowdisplayshortskip = 0pt plus 40pt minus 0pt
    \pagestyle{fancy}

    \subsection*{1.1 Spannung}

    \begin{equation}
      \boxed{
        \tag{1.1}
        \begin{split}
          \underbrace{\sigma}_{\text{Spannung}\left[\frac{N}{mm^2}\right]}
          =
          \frac{\overbrace{N}^{\text{Normalspannung}\left[N\right]}}{\underbrace{A}_{\text{Fläche}\left[mm^2\right]}}
        \end{split}
      }
    \end{equation}

    \begin{equation}
      \boxed{
        \tag{1.2}
        \begin{split}
          \underbrace{\sigma}_{\text{Spannung}\left[\frac{N}{mm^2}\right]}
          =
          \frac{\overbrace{F}^{\text{Kraft}\left[N\right]}}{\underbrace{A}_{\text{Fläche}\left[mm^2\right]}}
        \end{split}
      }
    \end{equation}

    \begin{equation}
      \boxed{
        \tag{1.3}
        \begin{split}
          \sigma
          =
          \frac{\overbraceWrap[95pt]{\sigma_0}{Normalspannung in einem Schnitt Senkrecht zur Stabachse}}{2}
          \left(1 + \cos 2 \phi\right)
          ,&
          \tau
          =
          \frac{\sigma_0}{2}
          \left(\sin 2 \phi\right)
        \end{split}
      }
    \end{equation}

    \begin{equation}
      \boxed{
        \tag{1.4}
        \begin{split}
          \sigma(x)
          =
          \frac{N(x)}{A(x)}
        \end{split}
      }
    \end{equation}

    \begin{equation}
      \boxed{
        \tag{1.5}
        \begin{split}
          A_{\text{erf}}
          =
          \frac{\abs{N}}{\sigma_{\text{zul}}}
        \end{split}
      }
    \end{equation}

    \subsection*{1.2 Dehnung}

    \begin{equation}
      \boxed{
        \tag{1.6}
        \begin{split}
          \underbrace{\varepsilon}_{\text{Dehnung} \left[1\right]} 
          =
          \frac{\overbrace{\Delta \ell}^{\text{Verlängerung}\left[m\right]}}
               {\underbraceWrap[50pt]{\ell_0}{Ursprüngliche Länge $\left[m\right]$}}
          =
          \frac{\ell - \ell_0}{\ell_0}
        \end{split}
      }
    \end{equation}


    Örtliche (lokale Dehnung)
    \begin{equation}
      \boxed{
        \tag{1.7}
        \begin{split}
          \varepsilon(x)
          =
          \frac{\text{d}u}
               {\text{d}x}
        \end{split}
      }
    \end{equation}


    \subsection*{1.3 Stoffgesetz}

    Hooke'sches Gesetz
    \begin{equation}
      \boxed{
        \tag{1.8}
        \begin{split}
          \underbraceWrap[60pt]{E}{Elastizitätsmodul $\left[\frac{N}{mm^2}\right]$}
          =
          \frac{\overbrace{\sigma}^{\text{Spannung}\left[\frac{N}{mm^2}\right]}}
               {\underbrace{\varepsilon}_{\text{Dehnung}\left[1\right]}}
        \end{split}
      }
    \end{equation}

    Umgestellt nach Sigma, übliche Form:
    \begin{equation*}
        \begin{split}
          \sigma
          =
          E \varepsilon
          =
          \frac{\Delta \ell}{\ell_0} E
        \end{split}
    \end{equation*}

    \begin{equation}
      \boxed{
        \tag{1.9}
        \begin{split}
          \underbrace{\varepsilon}_{\text{Dehnung} \left[1\right]}
          =
          \frac{\sigma}{E}
        \end{split}
      }
    \end{equation}

    \begin{equation}
      \boxed{
        \tag{1.10}
        \begin{split}
          \underbrace{\varepsilon_T}_{\text{Wärmedehnung} \left[1\right]}
          =
          \underbraceWrap[110pt]
            {\alpha}
            {
              Thermischer Ausdehnungskoeffizient (Wärmeausdehnugnskoeffizient)
              $\left[
                \nicefrac{1}
                         {\,^{\circ}\mathrm{C}}
              \right]$
            }
          \cdot
          \underbrace{\Delta T}_{\text{Temperaturänderung}\left[\,^{\circ}\mathrm{C}\right]}
        \end{split}
      }
    \end{equation}

    \begin{equation}
      \boxed{
        \tag{1.11}
        \begin{split}
          \varepsilon
          =
          \frac{\sigma}{E}
          +
          \alpha_T
          \Delta T
        \end{split}
      }
    \end{equation}
    
    \begin{equation}
      \boxed{
        \tag{1.12}
        \begin{split}
          \sigma
          =
          E
          \left(
            \varepsilon
            -
            \alpha_T
            \Delta T
          \right)
        \end{split}
      }
    \end{equation}


    \subsection*{1.4 Einzelstab}

    \begin{equation}
      \boxed{
        \tag{1.13}
        \begin{split}
          \frac{\text{d}N}
               {\text{d}x}
          +
          \underbrace{n}_{\text{Linienkraft}}
          =
          0
        \end{split}
      }
    \end{equation}

    \begin{equation}
      \boxed{
        \tag{1.14}
        \begin{split}
          \frac{\text{d}u}
               {\text{d}x}
          =
          \frac{N}{EA}
          +
          \alpha_T \Delta T
        \end{split}
      }
    \end{equation}

    \begin{equation}
      \boxed{
        \tag{1.15}
        \begin{split}
          \Delta \ell
          =
          u(l)
          -
          u(0)
          =
          \int_{0}^{\ell}
          \varepsilon
          \text{d}x
        \end{split}
      }
    \end{equation}
     
    \begin{equation}
      \boxed{
        \tag{1.16}
        \begin{split}
          \Delta \ell
          =
          \int_{0}^{\ell}
          \left(
            \frac{N}{EA}
            +
            \alpha_T \Delta T
          \right)
          \text{d}x
        \end{split}
      }
    \end{equation}

    \begin{equation}
      \boxed{
        \tag{1.17}
        \begin{split}
          \Delta \ell
          =
          \frac{F\ell}{EA}
          +
          \alpha_T \Delta T \ell
        \end{split}
      }
    \end{equation}


    Für $\Delta T = 0$
    \begin{equation}
      \boxed{
        \tag{1.18}
        \begin{split}
          \Delta \ell
          =
          \frac{F\ell}{EA}
        \end{split}
      }
    \end{equation}

    Oder $F = 0$
    \begin{equation}
      \boxed{
        \tag{1.19}
        \begin{split}
          \Delta \ell
          =
          \alpha_T \Delta T \ell
        \end{split}
      }
    \end{equation}
    
    \begin{equation}
      \boxed{
        \tag{1.20a}
        \begin{split}
          \left(
            EA u'
          \right)'
          =
          -n
          +
          \left(
            EA \alpha_t \Delta T
          \right)'
        \end{split}
      }
    \end{equation}

    Sei in 1.20a $EA = const$ und $\Delta T = const$
    \begin{equation}
      \boxed{
        \tag{1.20b}
        \begin{split}
          EA u''
          =
          -n
        \end{split}
      }
    \end{equation}


    \subsection*{1.5 Statisch bestimmte Stabsysteme}

    \begin{equation}
      \boxed{
        \tag{1.21}
        \begin{split}
          u
          &=
          \abs{\Delta\ell_1}
          =
          \frac{F\ell}{EA}
          \frac{1}{\tan{\alpha}}
          ,\\
          v
          &=
          \frac{\Delta\ell_2}{\sin{\alpha}}
          +
          \frac{u}{\tan{\alpha}}
          =
          \frac{F\ell}{EA}
          \frac{1 + \cos^3\alpha}{\sin^2\alpha \cos \alpha}
        \end{split}
      }
    \end{equation}


    \subsection*{1.6 Statisch unbestimmte Stabsysteme}

    \subsection*{1.7 Zusammenfassung}
    
    \subsection*{2.1 Spannungvektor und Spannungtensor}

    \begin{equation}
      \boxed{
        \tag{2.1}
        \begin{split}
          \boldsymbol{t}
          =
          \lim_{\Delta A \rightarrow 0}
          \frac{\Delta \boldsymbol{F}}{\Delta A}
          =
          \frac{\text{d}\boldsymbol{F}}{\text{d}A}
        \end{split}
      }
    \end{equation}

    \begin{equation}
      \boxed{
        \tag{2.2}
        \begin{split}
          \boldsymbol{t}
          =
          \tau_{yx} \boldsymbol{e_x}
          +
          \sigma_y  \boldsymbol{e_y}
          +
          \tau_{yz} \boldsymbol{e_z}
        \end{split}
      }
    \end{equation}

    \begin{equation}
      \boxed{
        \tag{2.3}
        \begin{split}
          \tau_{xy} = \tau_{yx},
          \tau_{xz} = \tau_{zx},
          \tau_{yz} = \tau_{zy}
        \end{split}
      }
    \end{equation}
    
    \begin{equation}
      \boxed{
        \tag{2.4}
        \begin{split}
          \boldsymbol{\sigma}
          =
          \begin{bmatrix*}
            \sigma_{x} & \tau_{xy}  & \tau_{xz} \\
            \tau_{yx}  & \sigma_{y} & \tau_{yz} \\
            \tau_{zx}  & \tau_{zy}  & \sigma_{z}
          \end{bmatrix*}
          =
          \begin{bmatrix*}
            \sigma_{x} & \tau_{xy}  & \tau_{xz} \\
            \tau_{xy}  & \sigma_{y} & \tau_{yz} \\
            \tau_{xz}  & \tau_{yz}  & \sigma_{z}
          \end{bmatrix*}          
        \end{split}
      }
    \end{equation}

    \begin{equation}
      \boxed{
        \tag{2.5a}
        \begin{split}
          \sigma_{\xi} &= \sigma_x \cos^2 \varphi + \sigma_y \sin^2\varphi + 2 \tau_{xy} \sin\varphi \cos\phi\\
          \tau_{\xi \eta} &= -(\sigma_x - \sigma_y) \sin \varphi \cos \varphi + \tau_{xy} (\cos^2\varphi - \sin^2\varphi)
        \end{split}
      }
    \end{equation}
   
    \begin{equation}
      \boxed{
        \tag{2.5b}
        \begin{split}
          \sigma_{\eta}
          =
          \sigma_{x} \sin^2 \varphi + \sigma_y \cos^2 \varphi - 2 \tau_{xy} \cos\varphi\sin\varphi
        \end{split}
      }
    \end{equation}

    \begin{equation}
      \boxed{
        \tag{2.6}
        \begin{split}
          \sigma_{\xi} &= \frac{1}{2} (\sigma_x + \sigma_y) +&\frac{1}{2}(\sigma_x - \sigma_y) \cos 2 \varphi + \tau_{xy} \sin 2 \varphi, \\
          \sigma_{\eta} &= \frac{1}{2} (\sigma_x + \sigma_y) -&\frac{1}{2}(\sigma_x - \sigma_y) \cos 2 \varphi + \tau_{xy} \sin 2 \varphi,\\
          \tau_{\xi \eta} &= -&\frac{1}{2}(\sigma_x - \sigma_y) \sin 2 \varphi + \tau_{xy} \cos 2 \varphi,
        \end{split}
      }
    \end{equation}
    
    \begin{equation}
      \boxed{
        \tag{2.7}
        \begin{split}
          \sigma_\xi + \sigma_\eta
          =
          \sigma_x + \sigma_y
        \end{split}
      }
    \end{equation}
    
    \begin{equation}
      \boxed{
        \tag{2.8}
        \begin{split}
          \tan 2\varphi^\ast
          =
          \frac{2 \tau_{xy}}{\sigma_x - \sigma_y}
        \end{split}
      }
    \end{equation}

    \begin{equation}
      \boxed{
        \tag{2.9}
        \begin{split}
          \cos 2\varphi^\ast
          &=
          \frac{1}{\sqrt{1+ \tan^2 2\varphi^\ast}}
          &=
          \frac{\sigma_x - \sigma_y}{\sqrt{(\sigma_x - \sigma_y)^2 + 4 \tau_{xy}^2}} \\
          \sin 2\varphi^\ast
          &=
          \frac{\tan 2\varphi^\ast}{\sqrt{1+ \tan^2 2\varphi^\ast}}
          &=
          \frac{2\tau_{xy}}{\sqrt{(\sigma_x - \sigma_y)^2 + 4 \tau_{xy}^2}} \\          
        \end{split}
      }
    \end{equation}

    \begin{equation}
      \boxed{
        \tag{2.10}
        \begin{split}
          \sigma_{1,2}
          =
          \frac{\sigma_x + \sigma_y}{2}
          \pm
          \sqrt{\left(
            \frac{\sigma_x - \sigma_y}{2}
            \right)^2
            +
            \tau_{xy}^2
          }
        \end{split}
      }
    \end{equation}

    \begin{equation}
      \boxed{
        \tag{2.11}
        \begin{split}
          \tan 2\varphi^{\ast\ast}
          =
          -\frac{\sigma_x - \sigma_y}{2 \tau_{xy}}
        \end{split}
      }
    \end{equation}

    \begin{equation}
      \boxed{
        \tag{2.12a}
        \begin{split}
          \tau_{\text{max}}
          =
          \pm
          \sqrt{(\frac{\sigma_x - \sigma_y}{2})^2 + \tau_{xy}^2}
        \end{split}
      }
    \end{equation}

    \begin{equation}
      \boxed{
        \tag{2.12b}
        \begin{split}
          \tau_{\text{max}}
          =
          \pm
          \frac{1}{2}(\sigma_1 -\sigma_2)
        \end{split}
      }
    \end{equation}

    \begin{equation}
      \boxed{
        \tag{2.13}
        \begin{split}
          \sigma_M
          =
          \frac{1}{2}(\sigma_x + \sigma_y)
          =
          \frac{1}{2}(\sigma_1 + \sigma_2)
        \end{split}
      }
    \end{equation}

\end{document}














































