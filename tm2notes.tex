\documentclass[11pt]{article}

\usepackage{amsmath,amssymb, a4, verbatim}
\usepackage[german]{babel}
%\usepackage[latin1]{inputenc}
\usepackage[utf8]{inputenc} % üöäß
\usepackage{listings} % für inline codelistings
\lstset{%
    basicstyle=\ttfamily,    % the size of the fonts 
    columns=fixed,            % anything else is horrifying
    showspaces=false,        % show spaces using underscores?
    showstringspaces=false,    % underline spaces within strings?
    showtabs=false,            % show tabs within strings?
    xleftmargin=1.5em,        % left margin space
}
\lstdefinestyle{inline}{basicstyle=\ttfamily}
\newcommand{\listline}[1]{\lstinline[style=inline]!#1!}


\usepackage{caption}
\newcommand{\tinycaption}[1]{\captionsetup{labelformat=empty}\caption{#1}}

%\usepackage{color}
%\usepackage{epsfig} % eps
\usepackage{graphicx} % eps
%\usepackage[shortcuts]{extdash}
%\usepackage{dsfont}
%\usepackage{epstopdf} % eps
%\usepackage[pdf]{pstricks} % eps
%\usepackage{auto-pst-pdf}
\usepackage{mathtools}
\usepackage{dsfont} % $ \mathds{1} $
\usepackage{icomma}
\usepackage{tikz}
%\usepackage{pgfplots}
%\pgfplotsset{compat=1.8}
\usepackage[bottom]{footmisc} % put footnotes at the bottom of page
\usepackage{nicefrac} % für brüche die aussehen wie prozentzeichen
% \usepackage{ps2pdf}
\usetikzlibrary{automata,positioning}

\usepackage{algorithmicx}
\usepackage{algpseudocode}
\usepackage{algorithm}

\usepackage{multicol}
\usepackage{wrapfig} % make stuff float
\usepackage{placeins} % stop stuff from floating
\usepackage{seqsplit} % very long numbers
\usepackage{framed} % begin{framed}

%  Headings and Footings :
\usepackage{fancyhdr}
\headheight15pt
\lhead{Technische Mechanik II, \ueberschrift}

\chead{}
\rhead{\thepage}
\renewcommand{\headrulewidth}{.4pt}

\lfoot{\today}
\cfoot{}
\rfoot{Joshua}
\renewcommand{\footrulewidth}{.4pt}

%----------------------------------------------------------------

\textwidth16.5cm
\oddsidemargin0.cm
\evensidemargin0.cm

\parindent0cm

\newcommand{\R}{ {\mathbb R} }
\newcommand{\C}{ {\mathbb C} }
\newcommand{\1}{ {\mathds{1}} }
\newcommand{\abs}[1]{\lvert#1\rvert}
\newcommand{\norm}[1]{\left\lVert#1\right\rVert}
\newcommand{\xt}{\tilde{x}}
\newcommand{\dotleq}{\dot{\leq}}
\newcommand{\m}{\hphantom{-} }

\newcommand{\dashfill}[1]{\vspace{11pt}\def\dashfill{\cleaders\hbox{#1}\hfill}\hbox to \hsize{\dashfill\hfil}\vspace{11pt}}
\newcommand{\scdot}{\!\cdot\!}


\newcommand{\sig}{\text{signum}}
\newcommand{\rot}{}

% ------------------  edit Ueberschrift ---------------------
\newcommand{\ueberschrift}{TM II Notizen}


% -----------------------------------------------------------
\begin{document}
    \abovedisplayskip = 5pt plus 40pt minus 0pt
    \belowdisplayskip = 5pt plus 40pt minus 0pt
    \abovedisplayshortskip = 0pt plus 40pt minus 0pt
    \belowdisplayshortskip = 0pt plus 40pt minus 0pt
    \pagestyle{fancy}
    \begin{align*}
        1 \text{Newton} \coloneqq \frac{kg \cdot \text{meter}}{\text{sekunde}^2}
    \end{align*}

    \hrulefill
    
    Spannung:
    \begin{align*}
        \underbrace{\sigma}_{\text{Spannung}\left[\frac{N}{mm^2}\right]}
        =
        \frac{\overbrace{F}^{\text{Kraft}\left[N\right]}}{\underbrace{A}_{\text{Fläche}\left[mm^2\right]}}
    \end{align*}

    \hrulefill
    \begin{align*}
        \underbrace{F_G}_{\text{Gewichtskraft} [N = \frac{kg m}{s^2}]} &= \underbrace{m}_{Masse\left[kg\right]} \cdot \underbrace{g}_{Fallbeschleunigung \left[\frac{m}{s^2}\right]} \\ 
        &= \underbrace{V}_{Volumen\left[m^3\right]} \cdot \underbrace{\rho}_{Dichte\left[\frac{kg}{m^3}\right]} \cdot g
    \end{align*}
    
    \hrulefill
    \begin{align*}
        \underbrace{\Delta \ell}_{\text{Verlängerung}\left[m\right]} = \underbrace{\ell}_{\text{belastete Länge} \left[m\right]} - \underbrace{\ell_0}_{\text{Urprungslänge} \left[m\right]}
    \end{align*}
    Die Verlängerung $\Delta \ell$ ist $>0$ wenn Das Teil länger wird, daran gezogen wird.\\
    Die Verlängerung $\Delta \ell$ ist $<0$ wenn Das Teil kürzer wird, daran gedrückt wird.

    \hrulefill
    \begin{align*}
        \underbrace{\varepsilon}_{\text{Dehnung} \left[Einheitslos \; \widehat{=}\; 1\right]} 
        =
        \frac{\overbrace{\Delta \ell}^{\text{Verlängerung}\left[m\right]}}{\underbrace{\ell_0}_{\text{Ursprungslänge}\left[m\right]}}
        = \frac{\ell - \ell_0}{\ell_0}
    \end{align*}
    $\varepsilon$ ist die Dehnung als relative Angabe, also in \%.

    \hrulefill
    
    Querdehnung, Änderung der Dicke durch Belastung normal dazu.
    \begin{align*}
        \underbrace{\varepsilon_q}_{\text{Querdehnung} \left[1\right]} 
        =
        \frac{\overbrace{\Delta d}^{\text{Dickenänderung}\left[m\right]}}{\underbrace{d_0}_{\text{Ursprüngliche Dicke} \left[m\right]}}
        = \frac{d - d_0}{d_0}
    \end{align*}    

    \hrulefill
    \begin{align*}
        \underbrace{m}_{\text{Poisson-Zahl}[1]} = \frac{\overbrace{\varepsilon}^{\text{Dehnung} [1]}}{\underbrace{\varepsilon_q}_{\text{Querdehnung}[1]}}
    \end{align*}
    Auch als Kehrwert genutzt:
        \begin{align*}
        \underbrace{\mu}_{\text{Querzahl oder Querkontraktionszahl}[1]} = \frac{1}{\underbrace{m}_{\text{Poisson-Zahl}[1]}}
    \end{align*}

    \hrulefill
        
    Hookesches Gesetz:
    \begin{align*}
        \underbrace{E}_{\text{Elastizitätsmodul}\left[\frac{N}{mm^2}\right]}
        =
        \frac{\overbrace{\sigma}^{\text{Spannung}\left[\frac{N}{mm^2}\right]}}
        {\underbrace{\varepsilon}_{\text{Dehnung}\left[1\right]}}
    \end{align*}
Umgestellt nach Sigma, übliche Form:
    \begin{align*}
        \sigma
        =
        \varepsilon E
        =
        \frac{\Delta \ell}{\ell_0}E
    \end{align*}
    \hrulefill

    Wärmespannung:
    \begin{align*}
        \underbrace{\Delta \ell}_{\left[ mm \right]}
        =
        \underbrace{\ell_0}_{\text{Ursprungslänge}\left[mm\right]} \cdot
        \underbrace{\alpha_{\ell}}_{\text{Längenausdehnungskoeffizient}\left[\frac{1}{K}\right]} \cdot
        \underbrace{\Delta T}_{\text{Temperaturunterschied}\left[K\right]} \cdot
    \end{align*}
    \hrulefill
    
    \pagebreak
    HA: abgesetzter Torsionsstab

    mit Festeinspannung Wand links.
    Dann Wand als $A$, mitte (verjünngung) als $B$, rechtes Ende als $C$.

    Durchmesser linkes Teil ist $d_1=60mm$, rechtes Teil $d_2=40mm$.

    Bei $B$ grieft ein Torsionsmoment $M_{t_B} = 3 kNm$ an. und am rechten Ende auch eines: $M_{t_C} = 0,6 kNm$.

    Abstand von $\overline{AB} = l_1 = 1m$, $\overline{BC} = l_2 = 1,5m$.

    $G = 0,8 \cdot 10^5 \nicefrac{N}{mm^2}$
    
    \hrulefill
    
\end{document}














































